%%%%%%%%%%%%%%%%%%%%%%%%%%%%%%%%%%%%%%%%%%%%%%%%%%%%%%%%%%%%%%%%%%%%%%
\section{\label{sec:Pool-Management}Pool Management}
%%%%%%%%%%%%%%%%%%%%%%%%%%%%%%%%%%%%%%%%%%%%%%%%%%%%%%%%%%%%%%%%%%%%%%
\index{pool management}

HTCondor provides administrative tools to help with
pool management.
This section describes some of these tasks.

All of the commands described in this section are subject to the
security policy chosen for the HTCondor pool.
As such, the commands must be either run from a
machine that has the proper authorization, 
or run by a user that is authorized to issue the commands.
Section~\ref{sec:Security} on
page~\pageref{sec:Security} details the implementation of 
security in HTCondor.

%%%%%%%%%%%%%%%%%%%%%%%%%%%%%%%%%%%%%%%%%%%%%%%%%%%%%%%%%%%%%%%%%%%%%%
\subsection{\label{sec:Pool-Upgrade}
Upgrading -- Installing a New Version on an Existing Pool}
%%%%%%%%%%%%%%%%%%%%%%%%%%%%%%%%%%%%%%%%%%%%%%%%%%%%%%%%%%%%%%%%%%%%%%
\index{pool management!installing a new version on an existing pool}
\index{installation!installing a new version on an existing pool}

An upgrade changes the running version of HTCondor
from the current installation to a newer version.
The safe method
to install and start running a newer version of HTCondor
in essence is:
shut down the current installation of HTCondor,
install the newer version,
and then restart HTCondor using the newer version.
To allow for falling back to the current version,
place the new version in a separate directory.
Copy the existing configuration files,
and modify the copy to point to and use the new version,
as well as incorporate any configuration variables that are new or changed
in the new version.
Set the \Env{CONDOR\_CONFIG} environment variable
to point to the new copy of the configuration,
so the new version of HTCondor will use the new configuration when restarted.

When upgrading from a version of HTCondor earlier than 6.8 to more recent version,
note that the configuration settings must be modified for security reasons.
Specifically, the \Macro{HOSTALLOW\_WRITE} configuration variable
must be explicitly changed,
or no jobs may be submitted,
and error messages will be issued by HTCondor tools.

Another way to upgrade leaves HTCondor running. 
HTCondor will automatically restart itself if the \Condor{master} binary
is updated,
and this method takes advantage of this. 
Download the newer version, placing it such that it does not 
overwrite the currently running version.
With the download will be a new set of configuration files;
update this new set with any specializations implemented in the currently
running version of HTCondor.
Then, modify the currently running installation by changing its
configuration such that the path to binaries points instead
to the new binaries.
One way to do that (under Unix) is to use a symbolic link that points 
to the current HTCondor installation directory (for example, \File{/opt/condor}).
Change the symbolic link to point to the new directory. 
If HTCondor is configured to locate its binaries via the symbolic link, 
then after the symbolic link changes, 
the \Condor{master} daemon notices the new binaries and restarts itself. 
How frequently it checks is controlled by the configuration variable 
\Macro{MASTER\_CHECK\_NEW\_EXEC\_INTERVAL}, which defaults 5 minutes.

When the \Condor{master} notices new binaries, 
it begins a graceful restart. 
On an execute machine, 
a graceful restart means that running jobs are preempted. 
Standard universe jobs will attempt to take a checkpoint. 
This could be a bottleneck if all machines in a large pool 
attempt to do this at the same time. 
If they do not complete within the cutoff time specified by the \MacroNI{KILL} 
policy expression (defaults to 10 minutes), 
then the jobs are killed without producing a checkpoint. 
It may be appropriate to increase this cutoff time, 
and a better approach may be to upgrade the pool in stages 
rather than all at once. 

For universes other than the standard universe, jobs are preempted. 
If jobs have been guaranteed a certain amount of uninterrupted run time 
with \MacroNI{MaxJobRetirementTime}, 
then the job is not killed until the specified amount of 
retirement time has been exceeded (which is 0 by default). 
The first step of killing the job is a soft kill signal, 
which can be intercepted by the job so that it can exit gracefully, 
perhaps saving its state. 
If the job has not gone away once the \MacroNI{KILL} expression fires 
(10 minutes by default), 
then the job is forcibly hard-killed. 
Since the graceful shutdown of jobs may rely on shared resources such as disks 
where state is saved, 
the same reasoning applies as for the standard universe: 
it may be appropriate to increase the cutoff time 
for large pools, 
and a better approach may be to upgrade the pool in stages 
to avoid jobs running out of time. 

Another time limit to be aware of is the configuration variable 
\MacroNI{SHUTDOWN\_GRACEFUL\_TIMEOUT}. 
This defaults to 30 minutes. 
If the graceful restart is not completed within this time, 
a fast restart ensues. 
This causes jobs to be hard-killed. 

%%%%%%%%%%%%%%%%%%%%%%%%%%%%%%%%%%%%%%%%%%%%%%%%%%%%%%%%%%%%%%%%%%%%%%
\subsection{\label{sec:Pool-Shutdown-and-Restart}
Shutting Down and Restarting an HTCondor Pool}
%%%%%%%%%%%%%%%%%%%%%%%%%%%%%%%%%%%%%%%%%%%%%%%%%%%%%%%%%%%%%%%%%%%%%%
\index{pool management!shutting down HTCondor}
\index{pool management!restarting HTCondor}

\begin{description}
\item[Shutting Down HTCondor]
There are a variety of ways to shut down all or parts of an HTCondor pool.
All utilize the \Condor{off} tool.

To stop a single execute machine from running jobs,
the \Condor{off} command specifies the machine by host name.
\begin{verbatim}
  condor_off -startd <hostname>
\end{verbatim}
A running \SubmitCmd{standard} universe job will be allowed to 
take a checkpoint before the job is killed.
A running job under another universe will be killed.
If it is instead desired that the machine stops running jobs
only after the currently executing job completes, the command is
\begin{verbatim}
  condor_off -startd -peaceful <hostname>
\end{verbatim}
Note that this waits indefinitely for the running job to finish,
before the \Condor{startd} daemon exits.

Th shut down all execution machines within the pool,
\begin{verbatim}
  condor_off -all -startd
\end{verbatim}

To wait indefinitely for each machine in the pool to finish its current
HTCondor job,
shutting down all of the execute machines as they no longer
have a running job,
\begin{verbatim}
  condor_off -all -startd -peaceful
\end{verbatim}

To shut down HTCondor on a machine from which jobs are submitted,
\begin{verbatim}
  condor_off -schedd <hostname>
\end{verbatim}

If it is instead desired that the submit machine shuts down
only after all jobs that are currently in the queue are finished,
first disable new submissions to the queue 
by setting the configuration variable
\begin{verbatim}
  MAX_JOBS_SUBMITTED = 0
\end{verbatim}
See instructions below in section~\ref{sec:Reconfigure-Pool} for how
to reconfigure a pool.
After the reconfiguration, the command to wait for all jobs to complete
and shut down the submission of jobs is
\begin{verbatim}
  condor_off -schedd -peaceful <hostname>
\end{verbatim}

Substitute the option \Opt{-all} for the host name,
if all submit machines in the pool are to be shut down.

\item[Restarting HTCondor, If HTCondor Daemons Are Not Running]
If HTCondor is not running,
perhaps because one of the \Condor{off} commands was used,
then starting HTCondor daemons back up depends on which part of
HTCondor is currently not running.

If no HTCondor daemons are running, then starting HTCondor is a matter
of executing the \Condor{master} daemon.
The \Condor{master} daemon will then invoke all other specified daemons
on that machine.
The \Condor{master} daemon executes on every machine that is to run HTCondor.

If a specific daemon needs to be started up, and the \Condor{master} daemon
is already running, then issue the command on the specific machine with
\begin{verbatim}
  condor_on -subsystem <subsystemname>
\end{verbatim}
where \verb@<subsystemname>@ is replaced by the daemon's subsystem
name.
Or, this command might be issued from another machine in the pool
(which has administrative authority) with
\begin{verbatim}
  condor_on <hostname> -subsystem <subsystemname>
\end{verbatim}
where \verb@<subsystemname>@ is replaced by the daemon's subsystem
name, and \verb@<hostname>@ is replaced by the host name of the
machine where this \Condor{on} command is to be directed.

\item[Restarting HTCondor, If HTCondor Daemons Are Running]
If HTCondor daemons are currently running, but need to be killed and
newly invoked,
the \Condor{restart} tool does this.
This would be the case for a new value of a configuration variable for
which using \Condor{reconfig} is inadequate.

To restart all daemons on all machines in the pool,
\begin{verbatim}
  condor_restart -all
\end{verbatim}

To restart all daemons on a single machine in the pool,
\begin{verbatim}
  condor_restart <hostname>
\end{verbatim}
where \verb@<hostname>@ is replaced by the host name of the
machine to be restarted.

\end{description}

%%%%%%%%%%%%%%%%%%%%%%%%%%%%%%%%%%%%%%%%%%%%%%%%%%%%%%%%%%%%%%%%%%%%%%
\subsection{\label{sec:Reconfigure-Pool}Reconfiguring an HTCondor Pool}
%%%%%%%%%%%%%%%%%%%%%%%%%%%%%%%%%%%%%%%%%%%%%%%%%%%%%%%%%%%%%%%%%%%%%%
\index{pool management!reconfiguration}

To change a global configuration variable and have all the
machines start to use the new setting, change the value within the file,
and send a \Condor{reconfig} command to each host.
Do this with a \emph{single} command,
\begin{verbatim}
  condor_reconfig -all
\end{verbatim}

If the global configuration file is not shared among all the machines,
as it will be if using a shared file system,
the change must be made to each copy of the global configuration file
before issuing the \Condor{reconfig} command.

Issuing a \Condor{reconfig} command is inadequate for some
configuration variables.
For those, a restart of HTCondor is required.
Those configuration variables that require a restart are listed in
section~\ref{sec:Macros-Requiring-Restart}
on page~\pageref{sec:Macros-Requiring-Restart}.
The manual page for \Condor{restart} is at
~\ref{man-condor-restart}.

\input{admin-man/absent.tex}
%%%%%%%%%%%%%%%%%%%%%%%%%%%%%%%%%%%%%%%%%%%%%%%%%%%%%%%%%%%%%%%%%%%%%%
\section{\label{sec:Monitoring}Monitoring}
%%%%%%%%%%%%%%%%%%%%%%%%%%%%%%%%%%%%%%%%%%%%%%%%%%%%%%%%%%%%%%%%%%%%%%
\index{pool management!monitoring}
\index{monitoring pools}
\index{pool monitoring}

Information that
the \Condor{collector} collects can be used to monitor a pool.
The \Condor{status} command can be used to display
snapshot of the current state of the pool.
Monitoring systems can be set up to track the state over time,
and they might go further, 
to alert the system administrator about exceptional conditions.


%%%%%%%%%%%%%%%%%%%%%%%%%%%%%%%%%%%%%%%%%%%%%%%%%%%%%%%%%%%%%%%%%%%%%%
\subsection{\label{sec:monitor-ganglia}Ganglia}
%%%%%%%%%%%%%%%%%%%%%%%%%%%%%%%%%%%%%%%%%%%%%%%%%%%%%%%%%%%%%%%%%%%%%%
\index{Monitoring!with Ganglia}
\index{Ganglia monitoring}
\index{daemon!condor\_gangliad@\Condor{gangliad}}
\index{condor\_gangliad daemon}

Support for the Ganglia monitoring system (\URL{http://ganglia.info/})
is integral to HTCondor.
Nagios (\URL{http://www.nagios.org/})
is often used to provide alerts based on data from the Ganglia
monitoring system.
The \Condor{gangliad} daemon provides an efficient way to take information from
an HTCondor pool and supply it to the Ganglia monitoring system.

The \Condor{gangliad} gathers up data as specified by its configuration,
and it streamlines getting that data to the Ganglia monitoring
system.
Updates sent to Ganglia are done using the Ganglia shared libraries for
efficiency.

If Ganglia is already deployed in the pool,
the monitoring of HTCondor is enabled by running the \Condor{gangliad} daemon
on a single machine within the pool.
If the machine chosen is the one running Ganglia's \Prog{gmetad},
then the HTCondor configuration consists of
adding \Expr{GANGLIAD} to the definition of configuration
variable \MacroNI{DAEMON\_LIST} on that machine.
It may be advantageous to run the \Condor{gangliad} daemon
on the same machine as is running the \Condor{collector} daemon,
because on a large pool with many ClassAds,
there is likely to be less network traffic.
If the \Condor{gangliad} daemon is to run on a different machine
than the one running Ganglia's \Prog{gmetad},
modify configuration variable \Macro{GANGLIA\_GSTAT\_COMMAND} to get the
list of monitored hosts from the master \Prog{gmond} program.

If the pool does not use Ganglia, 
the pool can still be monitored by a separate server running Ganglia.

By default, the \Condor{gangliad} will only propagate metrics to hosts
that are already monitored by Ganglia.
Set configuration variable \Macro{GANGLIA\_SEND\_DATA\_FOR\_ALL\_HOSTS} 
to \Expr{True} to set up
a Ganglia host to monitor a pool not monitored by Ganglia
or have a heterogeneous pool where some hosts are not monitored.
In this case, default graphs that Ganglia provides will not be present.
However, the HTCondor metrics will appear.

On large pools, 
setting configuration variable \Macro{GANGLIAD\_PER\_EXECUTE\_NODE\_METRICS}
to \Expr{False} will reduce the amount of data sent to Ganglia.
The execute node data is the least important to monitor.
One can also limit the amount of data by setting configuration variable
\Macro{GANGLIAD\_REQUIREMENTS}.
Be aware that aggregate sums over the entire pool will not be accurate
if this variable limits the ClassAds queried.

Metrics to be sent to Ganglia are specified in all files within the
directory specified by configuration variable 
\Macro{GANGLIAD\_METRICS\_CONFIG\_DIR}.
Each file in the directory is read,
and the format within each file is that of New ClassAds.
Here is an example of a single metric definition given as a New ClassAd:

\begin{verbatim}
[
  Name   = "JobsSubmitted";
  Desc   = "Number of jobs submitted";
  Units  = "jobs";
  TargetType = "Scheduler";
]
\end{verbatim}

A nice set of default metrics is in file:
\File{\$(GANGLIAD\_METRICS\_CONFIG\_DIR)/00\_default\_metrics}.

Recognized metric attribute names and their use:

  \begin{description}

  \item[Name] The name of this metric,  
    which corresponds to the ClassAd attribute name.
    Metrics published for the same machine must have unique names.

  \item[Value] A ClassAd expression that produces the value when evaluated.
    The default value is the value in the daemon ClassAd of the
    attribute with the same name as this metric.

  \item[Desc] A brief description of the metric.  This string is displayed 
    when the user holds the mouse over the Ganglia graph for the metric.

  \item[Verbosity] The integer verbosity level of this metric.  
    Metrics with a higher verbosity level than that specified by
    configuration variable \Macro{GANGLIA\_VERBOSITY} will not be published.

  \item[TargetType] A string containing a comma-separated list of daemon
    ClassAd types that this metric monitors.  The specified values should
    match the value of \Attr{MyType} of the daemon ClassAd.  
    In addition, there are
    special values that may be included. \verb|"Machine_slot1"| may be
    specified to monitor the machine ClassAd for slot 1 only.  This is
    useful when monitoring machine-wide attributes.  The special
    value \verb|"ANY"| matches any type of ClassAd.

  \item[Requirements] A boolean expression that may restrict how this
    metric is incorporated.  It defaults to \Expr{True}, which places
    no restrictions on the collection of this ClassAd metric.

  \item[Title] The graph title used for this metric.  The default is the
    metric name.

  \item[Group] A string specifying the name of this metric's group.
    Metrics are arranged by group within a Ganglia web page.  The
    default is determined by the daemon type.  Metrics in different
    groups must have unique names.

  \item[Cluster] A string specifying the cluster name for this metric.
    The default cluster name is taken from the configuration variable
    \Macro{GANGLIAD\_DEFAULT\_CLUSTER}.

  \item[Units] A string describing the units of this metric.

  \item[Scale] A scaling factor that is multiplied by the value of the
    \Attr{Value} attribute.
    The scale factor is used when the value is not in the basic unit
    or a human-interpretable unit. For example, duty cycle is commonly
    expressed as a percent, but the HTCondor value ranges from 0 to 1.
    So, duty cycle is scaled by 100. Some metrics are reported in Kbytes.
    Scaling by 1024 allows Ganglia to pick the appropriate units,
    such as number of bytes rather than number of Kbytes. 
    When scaling by large values, converting to
    the \verb|"float"| type is recommended.

  \item[Derivative] A boolean value that specifies if Ganglia should
    graph the derivative of this metric.  Ganglia versions prior to
    3.4 do not support this.

  \item[Type] A string specifying the type of the metric.  Possible
    values are \verb|"double"|, \verb|"float"|, \verb|"int32"|,
    \verb|"uint32"|, \verb|"int16"|, \verb|"uint16"|,
    \verb|"int8"|, \verb|"uint8"|, and \verb|"string"|.
    The default is \verb|"string"| for string values,
    The default is \verb|"int32"| for integer values,
    The default is \verb|"float"| for real values,
    The default is \verb|"int8"| for boolean values.
    Integer values can be coerced to \verb|"float"| or \verb|"double"|.
    This is especially important for values stored internally as 64-bit
    values.

  \item[Regex] This string value specifies a regular expression that
    matches attributes to be monitored by this metric.  This is useful
    for dynamic attributes that cannot be enumerated in advance,
    because their names depend on dynamic information such as the
    users who are currently running jobs.  When this is specified, one
    metric per matching attribute is created.  The default metric name
    is the name of the matched attribute, and the default value is the
    value of that attribute.  As usual, the \Attr{Value} expression
    may be used when the raw attribute value needs to be manipulated
    before publication.  However, since the name of the attribute is
    not known in advance, a special ClassAd attribute in the daemon ClassAd
    is provided to allow the \Attr{Value} expression to refer to it.
    This special attribute is named \Attr{Regex}.  Another special
    feature is the ability to refer to text matched by regular
    expression groups defined by parentheses within the regular
    expression.  These may be substituted into the values of other
    string attributes such as \Attr{Name} and \Attr{Desc}.  This is
    done by putting macros in the string values.  \verb|"\\1"| is
    replaced by the first group, \verb|"\\2"| by the second group, and
    so on.

  \item[Aggregate] This string value specifies an aggregation function
    to apply, instead of publishing individual metrics for each daemon
    ClassAd.  Possible values are \verb|"sum"|, \verb|"avg"|, \verb|"max"|,
    and \verb|"min"|.

  \item[AggregateGroup] When an aggregate function has been specified,
    this string value specifies which aggregation group the current
    daemon ClassAd belongs to.  The default is the metric \Attr{Name}.  This
    feature works like GROUP BY in SQL.  The aggregation function
    produces one result per value of \Attr{AggregateGroup}.  A single
    aggregate group would therefore be appropriate for a pool-wide
    metric.  Example use of aggregate grouping: if one wished to
    publish the sum of an attribute across different types of slot
    ClassAds, one could make the metric name an expression that is unique
    to each type.  The default \Attr{AggregateGroup} would be set
    accordingly.  Note that the assumption is still that the result
    is a pool-wide metric, so by default it is associated with the
    \Condor{collector} daemon's host.
    To group by machine and publish the result into
    the Ganglia page associated with each machine, one should make
    the \Attr{AggregateGroup} contain the machine name and override
    the default \Attr{Machine} attribute to be the daemon's machine
    name, rather than the \Condor{collector} daemon's.

  \item[Machine] The name of the host associated with this metric.  
    If configuration variable
    \Macro{GANGLIAD\_DEFAULT\_MACHINE} is not specified, 
    the default
    is taken from the \Attr{Machine} attribute of the daemon ClassAd.
    If the daemon name is of the form \verb|name@hostname|, this may
    indicate that there are multiple instances of HTCondor running on
    the same machine.  To avoid the metrics from these instances
    overwriting each other, the default machine name is set to the
    daemon name in this case.  For aggregate metrics, the default
    value of \Attr{Machine} will be the name of the \Condor{collector} host.

  \item[IP] A string containing the IP address of the host associated
    with this metric.  If \Macro{GANGLIAD\_DEFAULT\_IP} is not
    specified, the default is extracted from the \Attr{MyAddress}
    attribute of the daemon ClassAd.  This value must be unique for each
    machine published to Ganglia.  It need not be a valid IP address.
    If the value of \Attr{Machine} contains an \verb|"@"| sign, the
    default IP value will be set to the same value as \Attr{Machine}
    in order to make the IP value unique to each instance of HTCondor
    running on the same host.

  \end{description}

\input{admin-man/absent.tex}

