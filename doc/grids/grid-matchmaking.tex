%%%%%%%%%%%%%%%%%%%%%%%%%%%%%%%%%%%%%%%%%%%%%%%%%%
\subsection{\label{sec:Grid-Matchmaking}Matchmaking in the Grid Universe}
%%%%%%%%%%%%%%%%%%%%%%%%%%%%%%%%%%%%%%%%%%%%%%%%%%
\index{matchmaking!on the Grid}
\index{grid computing!matchmaking}

In a simple usage, the grid universe allows users to specify a single
grid site as a destination for jobs.
This is sufficient when a user knows exactly which
grid site they wish to use,
or a higher-level resource broker
(such as the European Data Grid's resource broker)
has decided which grid site should be used.

When a user has a variety of grid sites to choose from,
HTCondor allows matchmaking of grid universe jobs
to decide which grid resource a job should run on. 
Please note that this form of matchmaking is relatively new.
There are some rough edges as continual improvement occurs.

To facilitate HTCondor's matching of jobs with grid resources,
both the jobs and the grid resources are involved.
The job's submit description file provides all commands
needed to make the
job work on a matched grid resource.
The grid resource identifies itself to HTCondor by advertising
a ClassAd.
This ClassAd specifies all necessary attributes, such that HTCondor
can properly make matches.
The grid resource identification is accomplished by 
using \Condor{advertise} to send a ClassAd representing the
grid resource, which is then used by HTCondor to make matches.

%%%%%%%%%%%%%%%%%%%%%%%%%%%%%%%%%%%%%%%%%%%%%%%%%%
\subsubsection{Job Submission}
%%%%%%%%%%%%%%%%%%%%%%%%%%%%%%%%%%%%%%%%%%%%%%%%%%

To submit a grid universe job intended for a single, specific
\SubmitCmd{gt2} resource,
the submit description file for the job explicitly specifies
the resource:

\footnotesize
\begin{verbatim}
grid_resource = gt2 grid.example.com/jobmanager-pbs
\end{verbatim}
\normalsize

If there were multiple \SubmitCmd{gt2} resources that
might be matched to the job,
the submit description file changes:

\footnotesize
\begin{verbatim}
grid_resource   = $$(resource_name)
requirements    = TARGET.resource_name =!= UNDEFINED
\end{verbatim}
\normalsize

The \SubmitCmd{grid\_resource} command uses a substitution
macro.
The substitution macro defines the value 
of \Attr{resource\_name} using attributes
as specified by the matched grid resource.
The \SubmitCmd{requirements} command further restricts that
the job may only run on a machine (grid resource) that
defines \Attr{grid\_resource}.
Note that this attribute name is invented for this example.
To make matchmaking work in this way,
both the job (as used here within the submit description file)
and the grid resource (in its created and advertised ClassAd)
must agree upon the name of the attribute.

As a more complex example,
consider a job that wants
to run not only
on a \SubmitCmd{gt2} resource,
but on one that has the Bamboozle software installed.
The complete submit description file might appear:

\footnotesize
\begin{verbatim}
universe        = grid
executable      = analyze_bamboozle_data
output          = aaa.$(Cluster).out
error           = aaa.$(Cluster).err
log             = aaa.log
grid_resource   = $$(resource_name)
requirements    = (TARGET.HaveBamboozle == True) && (TARGET.resource_name =!= UNDEFINED)
queue
\end{verbatim}
\normalsize

Any grid resource which has the
\Attr{HaveBamboozle} attribute defined as well as
set to \Expr{True} is further checked to have the
\Attr{resource\_name} attribute defined.
Where this occurs, a match may be made (from the
job's point of view).
A grid resource that has one of these attributes defined,
but not the other results in no match being made.

Note that the entire value of \SubmitCmd{grid\_resource} comes from
the grid resource's ad. This means that the job can be matched with
a resource of any type, not just \SubmitCmd{gt2}.

%%%%%%%%%%%%%%%%%%%%%%%%%%%%%%%%%%%%%%%%%%%%%%%%%%
\subsubsection{Advertising Grid Resources to HTCondor}
%%%%%%%%%%%%%%%%%%%%%%%%%%%%%%%%%%%%%%%%%%%%%%%%%%

Any grid resource that wishes to be matched by HTCondor with
a job must advertise itself to HTCondor using a ClassAd.
To properly advertise, a ClassAd is sent
periodically to the \Condor{collector} daemon.
A ClassAd is a list of pairs, where each pair consists of
an attribute name and value that describes an entity.
There are two entities relevant to HTCondor:
a job, and a machine.
A grid resource is a machine.
The ClassAd describes the grid resource, as well
as identifying the capabilities of the grid resource.
It may also state both requirements and preferences
(called \SubmitCmd{rank}) for the jobs it will run.
See
Section~\ref{sec:matchmaking-with-classads} for an overview
of the interaction between matchmaking and ClassAds.
A list of common machine ClassAd attributes is given in
the Appendix on page~\pageref{sec:Machine-ClassAd-Attributes}.

To advertise a grid site, place the attributes
in a file.
Here is a sample ClassAd that describes a grid resource
that is capable of running a
\SubmitCmd{gt2} job.

\footnotesize
\begin{verbatim}
# example grid resource ClassAd for a gt2 job
MyType         = "Machine"
TargetType     = "Job"
Name           = "Example1_Gatekeeper"
Machine        = "Example1_Gatekeeper"
resource_name  = "gt2 grid.example.com/jobmanager-pbs"
UpdateSequenceNumber  = 4
Requirements   = (TARGET.JobUniverse == 9)
Rank           = 0.000000
CurrentRank    = 0.000000
\end{verbatim}
\normalsize

%% JobUniverse = 9 is the grid universe

Some attributes are defined as expressions, while
others are integers, floating point values, or strings.
The type is important, and must be correct for the
ClassAd to be effective.
The attributes
\begin{verbatim}
MyType         = "Machine"
TargetType     = "Job"
\end{verbatim}
identify the grid resource as a machine,
and that the machine is to be matched with a job.
In HTCondor, machines are matched with jobs, and jobs are matched with
machines.
These attributes are strings.
Strings are surrounded by double quote marks.

The attributes \Attr{Name} and \Attr{Machine}
are likely to be defined to be the same string value as in the
example:
\footnotesize
\begin{verbatim}
Name           = "Example1_Gatekeeper"
Machine        = "Example1_Gatekeeper"
\end{verbatim}
\normalsize

Both give the fully qualified host name for the resource.
The \Attr{Name} may be different on an SMP machine,
where the individual CPUs are given names that can
be distinguished from each other.
Each separate grid resource must have a unique name.

Where the job depends on the resource to specify the
value of the \SubmitCmd{grid\_resource} command by
the use of the substitution macro,
the ClassAd for the grid resource (machine)
defines this value.
The example given as
\footnotesize
\begin{verbatim}
grid_resource = "gt2 grid.example.com/jobmanager-pbs"
\end{verbatim}
\normalsize
defines this value.
Note that the invented name of this variable must match the
one utilized within the submit description file.
To make the matchmaking work,
both the job (as used within the submit description file)
and the grid resource (in this created and advertised ClassAd)
must agree upon the name of the attribute.

A machine's ClassAd information can be time sensitive,
and may change over time.
Therefore, ClassAds expire and are thrown away.
In addition, the communication method by which ClassAds
are sent implies that entire ads may be lost without notice
or may arrive out of order.
Out of order arrival leads to the definition of an
attribute which provides an ordering.
This positive integer value is given in the example
ClassAd as
\footnotesize
\begin{verbatim}
UpdateSequenceNumber  = 4
\end{verbatim}
\normalsize
This value must increase for each subsequent ClassAd.
If state information for the ClassAd is kept in a file,
a script executed each time the ClassAd is to be sent
may use a counter for this value.
An alternative for a stateless implementation sends
the current time in seconds (since the epoch, as given by
the C \Procedure{time} function call).

The requirements that the grid resource sets for any job
that it will accept are given as
\footnotesize
\begin{verbatim}
Requirements     = (TARGET.JobUniverse == 9)
\end{verbatim}
\normalsize
This set of requirements state that any job is
required to be for the \SubmitCmd{grid} universe.

The attributes
\footnotesize
\begin{verbatim}
Rank             = 0.000000
CurrentRank      = 0.000000
\end{verbatim}
\normalsize
are both necessary for HTCondor's negotiation to proceed,
but are not relevant to grid matchmaking.
Set both to the floating point value 0.0.


The example machine ClassAd becomes more complex
for the case where the grid resource allows matches with more
than one job:
\footnotesize
\begin{verbatim}
# example grid resource ClassAd for a gt2 job
MyType         = "Machine"
TargetType     = "Job"
Name           = "Example1_Gatekeeper"
Machine        = "Example1_Gatekeeper"
resource_name  = "gt2 grid.example.com/jobmanager-pbs"
UpdateSequenceNumber  = 4
Requirements   = (CurMatches < 10) && (TARGET.JobUniverse == 9)
Rank           = 0.000000
CurrentRank    = 0.000000
WantAdRevaluate = True
CurMatches     = 1
\end{verbatim}
\normalsize

In this example, the two attributes \Attr{WantAdRevaluate}
and \Attr{CurMatches} appear, and the \Attr{Requirements}
expression has changed.

\Attr{WantAdRevaluate} is a boolean value, and may be set to
either \Expr{True} or \Expr{False}.
When \Expr{True} in the ClassAd and a match is made (of a job
to the grid resource), the machine (grid resource)
is not removed from the set of machines to be considered for
further matches.
This implements the ability for a single grid resource to
be matched to more than one job at a time.
Note that the spelling of this attribute is incorrect,
and remains incorrect to maintain backward compatibility.

To limit the number of matches made to the single grid resource,
the resource must have the ability to keep track of the number 
of HTCondor jobs it has.
This integer value is given as the \Attr{CurMatches} attribute
in the advertised ClassAd.
It is then compared in order to limit the number of jobs matched
with the grid resource.
\footnotesize
\begin{verbatim}
Requirements   = (CurMatches < 10) && (TARGET.JobUniverse == 9)
CurMatches     = 1
\end{verbatim}
\normalsize

This example assumes that the grid resource already has
one job, and is willing to accept a maximum of 9 jobs.
If \Attr{CurMatches} does not appear in the ClassAd,
HTCondor uses a default value of 0.

\MacroIndex{NEGOTIATOR\_MATCHLIST\_CACHING}
\MacroIndex{NEGOTIATOR\_IGNORE\_USER\_PRIORITIES}
For multiple matching of a site ClassAd to work correctly,
it is also necessary to add the following to the configuration file
read by the \Condor{negotiator}:

\begin{verbatim}
NEGOTIATOR_MATCHLIST_CACHING = False
NEGOTIATOR_IGNORE_USER_PRIORITIES = True
\end{verbatim}

% COMMENTED OUT, AS THERE IS NOTHING ANYWHERE THAT SAYS HOW TO DO THIS!
% As an aside, we recommend that jobs that need specific applications
% should bring them with them instead of relying on having them
% pre-installed at a Grid site. You will have more reliable execution if
% you do. 

This ClassAd (likely in a file)
is to be periodically sent to the \Condor{collector} daemon
using \Condor{advertise}.
A recommended implementation uses a script to create or modify
the ClassAd together with
\Prog{cron} to send the ClassAd every five minutes.
The \Condor{advertise} program must be installed on the 
machine sending the ClassAd, but the remainder of HTCondor
does not need to be installed.
The required argument for the \Condor{advertise} command
is \Arg{UPDATE\_STARTD\_AD}.

\Condor{advertise} uses UDP to transmit the ClassAd.
Where this is insufficient,
specify the \Opt{-tcp} option to \Condor{advertise}
to use TCP for communication. 

%%%%%%%%%%%%%%%%%%%%%%%%%%%%%%%%%%%%%%%%%%%%%%%%%%
\subsubsection{Advanced usage}
%%%%%%%%%%%%%%%%%%%%%%%%%%%%%%%%%%%%%%%%%%%%%%%%%%

What if a job fails to run at a grid site due to an error? It will be
returned to the queue, and HTCondor will attempt to match it and
re-run it at another site. HTCondor isn't very clever about avoiding
sites that may be bad, but you can give it some assistance. Let's say
that you want to avoid running at the last grid site you ran at. You
could add this to your job description:

\footnotesize
\begin{verbatim}
match_list_length = 1
Rank              = TARGET.Name != LastMatchName0
\end{verbatim}
\normalsize

This will prefer to run at a grid site that was not just tried, but it
will allow the job to be run there if there is no other option. 

When you specify \Opt{match\_list\_length}, you provide an integer N, and
HTCondor will keep track of the last N matches. The oldest match will be
LastMatchName0, and next oldest will be LastMatchName1, and so on. (See
the \Condor{submit} manual page for more details.) The Rank expression
allows you to specify a numerical ranking for different matches. When
combined with \Opt{match\_list\_length}, you can prefer to avoid sites that
you have already run at. 

In addition, \Condor{submit} has two options to help control
grid universe job resubmissions and rematching.  
See the definitions of the submit description file commands
\Opt{globus\_resubmit} and \Opt{globus\_rematch} at 
page \pageref{condor-submit-globus-rematch} and
page \pageref{condor-submit-globus-resubmit}.
These options are independent of \Opt{match\_list\_length}.

There are some new attributes that will be added to the Job ClassAd,
and may be useful to you when you write your rank, requirements,
globus\_resubmit or globus\_rematch option. Please refer to
the Appendix on page~\pageref{sec:Job-ClassAd-Attributes} 
to see a list containing the following attributes:

\begin{itemize}
\item NumJobMatches
\item NumGlobusSubmits
\item NumSystemHolds
\item HoldReason
\item ReleaseReason
\item EnteredCurrentStatus
\item LastMatchTime
\item LastRejMatchTime
\item LastRejMatchReason
\end{itemize}

The following example of a command within the submit description file
releases jobs 5 minutes after being held,
increasing the time between releases by 5 minutes each time.
It will continue to retry up to 4 times per Globus
submission, plus 4.
The plus 4 is necessary in case
the job goes on hold before being submitted to Globus, although
this is unlikely.

\footnotesize
\begin{verbatim}
periodic_release = ( NumSystemHolds <= ((NumGlobusSubmits * 4) + 4) ) \
   && (NumGlobusSubmits < 4) && \
   ( HoldReason != "via condor_hold (by user $ENV(USER))" ) && \
   ((time() - EnteredCurrentStatus) > ( NumSystemHolds *60*5 ))
\end{verbatim}
\normalsize

The following example forces Globus resubmission after a job has
been held 4 times per Globus submission.

\footnotesize
\begin{verbatim}
globus_resubmit = NumSystemHolds == (NumGlobusSubmits + 1) * 4
\end{verbatim}
\normalsize

If you are concerned about unknown or malicious grid sites reporting
to your \Condor{collector}, you should use HTCondor's security options,
documented in Section~\ref{sec:Security}.
