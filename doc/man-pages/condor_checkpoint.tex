\begin{ManPage}{\label{man-condor-checkpoint}\Condor{checkpoint}}{1}
{send a checkpoint command to jobs running on specified hosts}
\index{HTCondor commands!condor\_checkpoint}
\index{condor\_checkpoint command}

\Synopsis \SynProg{\Condor{checkpoint}}
\ToolArgsBase

\SynProg{\Condor{checkpoint}}
\ToolDebugOption
\ToolWhere

\Description
\Condor{checkpoint} sends a checkpoint command to a set
of machines within a single pool.
This causes the startd daemon on each of the specified machines
to take a checkpoint of any running job that is executing under
the standard universe.
The job is temporarily stopped, a checkpoint is taken,
and then the job continues.
If no machine is specified, then the command
is sent to the machine that issued the
\Condor{checkpoint} command.

The command sent is a periodic checkpoint.
The job will take a checkpoint, but then the job will immediately
continue running after the checkpoint is completed.
\Condor{vacate}, on the other hand, will result in the job exiting
(vacating) after it produces a checkpoint. 

If the job being checkpointed is running under the standard universe,
the job produces a checkpoint and then continues running
on the same machine.
If the job is running under another universe,
or if there is currently no HTCondor job
running on that host, then \Condor{checkpoint} has no effect. 

There is generally no need for the user or administrator to explicitly
run \Condor{checkpoint}.
Taking checkpoints of running HTCondor jobs is
handled automatically following the policies
stated in the configuration files. 

\begin{Options}
	\ToolArgsBaseDesc
	\ToolDebugDesc
	\ToolArgsLocateDesc
\end{Options}

\ExitStatus

\Condor{checkpoint} will exit with a status value of 0 (zero) upon success,
and it will exit with the value 1 (one) upon failure.

\Examples
To send a \Condor{checkpoint} command to two named machines:
\begin{verbatim}
% condor_checkpoint  robin cardinal
\end{verbatim}

To send the \Condor{checkpoint} command to a machine
within a pool of machines other than the local pool,
use the \Opt{-pool} option.
The argument is the name of the central manager for the pool.
Note that one or more machines within the pool must be
specified as the targets for the command.
This command sends the command to
a the single machine named \Opt{cae17} within the
pool of machines that has \Opt{condor.cae.wisc.edu} as
its central manager:
\begin{verbatim}
% condor_checkpoint -pool condor.cae.wisc.edu -name cae17
\end{verbatim}

\end{ManPage}
