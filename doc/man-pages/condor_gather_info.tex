\begin{ManPage}{\label{man-condor-gather-info}\Condor{gather\_info}}{1}
{Gather information about an HTCondor installation and a queued job}
\Synopsis

\SynProg{\Condor{gather\_info}}
[\verb@--@\textbf{jobid} \textit{ClusterId.ProcId}]
[\verb@--@\textbf{scratch} \textit{/path/to/directory}]

% \index{HTCondor commands!condor\_gather\_info}
% \index{condor\_gather\_info command}

\Description

\Condor{gather\_info} is a Linux-only tool that
will collect and output information 
about the machine it is run upon,
about the HTCondor installation local to the machine, 
and optionally about a specified HTCondor job. 
The information gathered by this tool is most often used as a debugging aid
for the developers of HTCondor.

Without the \verb@--@\textbf{jobid} option, information about the
local machine and its HTCondor installation is gathered and
placed into the file called \File{condor-profile.txt},
in the current working directory. 
The information gathered is under the category of Identity.

With the \verb@--@\textbf{jobid} option, 
additional information is gathered about the job given
in the command line argument and identified by
its \Attr{ClusterId} and \Attr{ProcId} ClassAd attributes.
The information includes both categories, 
Identity and Job information.
As the quantity of information can be extensive,
this information is placed into a compressed tar file.
The file is placed into the current working directory,
and it is named using the format
\footnotesize
\begin{verbatim}
cgi-<username>-jid<ClusterId>.<ProcId>-<year>-<month>-<day>-<hour>_<minute>_<second>-<TZ>.tar.gz
\end{verbatim}
\normalsize
All values within \verb@<>@ are substituted with current values.
The building of this potentially large tar file can require a fair
amount of temporary space.
If the \verb@--@\textbf{scratch} option is specified,
it identifies a directory in which to build the tar file.
If the \verb@--@\textbf{scratch} option is \emph{not} specified, 
then the directory will be \File{/tmp/cgi-<PID>},
where the process ID is that of the \Condor{gather\_info} executable.

The information gathered by this tool:
\begin{enumerate}
	\item Identity
	\begin{itemize}
          \item User name who generated the report
          \item Script location and machine name
          \item Date of report creation
          \item \Shell{uname -a}
          \item Contents of \File{/etc/issue}
          \item Contents of \File{/etc/redhat-release}
          \item Contents of \File{/etc/debian\_version}
          \item Contents of \File{\$(LOG)/MasterLog}
          \item Contents of \File{\$(LOG)/ShadowLog}
          \item Contents of \File{\$(LOG)/SchedLog}
          \item Output of \Shell{ps -auxww --forest}
          \item Output of \Shell{df -h}
          \item Output of \Shell{iptables -L}
          \item Output of \Shell{ls `condor\_config\_val LOG`}
          \item Output of \Shell{ldd `condor\_config\_val SBIN`/condor\_schedd}
          \item Contents of \File{/etc/hosts}
          \item Contents of \File{/etc/nsswitch.conf}
          \item Output of \Shell{ulimit -a}
          \item Output of \Shell{uptime}
          \item Output of \Shell{free}
          \item Network interface configuration (\Shell{ifconfig})
          \item HTCondor version
          \item Location of HTCondor configuration files
          \item HTCondor configuration variables
		  \begin{itemize}
                \item All variables and values
                \item Definition locations for each configuration variable 
		  \end{itemize}
	\end{itemize}
	\item Job Information
	\begin{itemize}
    	\item Output of \Shell{condor\_q jobid}
    	\item Output of \Shell{condor\_q -l jobid}
    	\item Output of \Shell{condor\_q -analyze jobid}
    	\item Job event log, if it exists
		\begin{itemize}
          	\item Only events pertaining to the job ID 
		\end{itemize}
    	\item If \Condor{gather\_info} has the proper permissions,
it runs \Condor{fetchlog} on the machine where the job most recently ran,
and includes the contents of the logs from the \Condor{master},
\Condor{startd}, and \Condor{starter}.
	\end{itemize}
\end{enumerate}

\begin{Options}
  \OptItem{\OptArg{---jobid}{$<$ClusterId.ProcId$>$}}
  {Data mine information about this HTCondor job from the local 
  HTCondor installation and \Condor{schedd}.}
  \OptItem{\OptArg{---scratch}{/path/to/directory}}
  {A path to temporary space needed when building the output tar file.
  Defaults to \File{/tmp/cgi-<PID>}, where \Expr{<PID>} is replaced by
  the process ID of \Condor{gather\_info}.}
\end{Options}

\Files

\begin{itemize}
  \item{\File{condor-profile.txt}} The Identity portion of the information 
  gathered when \Condor{gather\_info} is run without arguments.

  \item{\File{cgi-<username>-jid<cluster>.<proc>-<year>-<month>-<day>-<hour>\_<minute>\_<second>-<TZ>.tar.gz}} The output file which contains all of 
  the information produced by this tool.
\end{itemize}

\ExitStatus

\Condor{gather\_info} will exit with a status value of 0 (zero) upon success,
and it will exit with the value 1 (one) upon failure.

\end{ManPage}
