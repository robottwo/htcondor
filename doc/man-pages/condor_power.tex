\begin{ManPage}{\label{man-condor-power}\Condor{power}}{1}
{send packet intended to wake a machine from a low power state}
\index{HTCondor commands!condor\_power}
\index{condor\_power command}

\Synopsis 
\SynProg{\Condor{power}}
\oOpt{-h}

\SynProg{\Condor{power}}
\oOpt{-d}
\oOpt{-i}
\oOptArg{-m}{MACaddress}
\oOptArg{-s}{subnet}
%\oOptArg{-p}{port}
\oArg{ClassAdFile}


\Description

\Condor{power} sends one UDP Wake on LAN (WOL) packet to a machine
specified either by command line arguments or by the contents
of a machine ClassAd.
The machine ClassAd may be in a file, where the 
file name specified by the optional argument \Arg{ClassAdFile} 
is given on the command line.
With no command line arguments to specify the machine,
and no file specified, \Condor{power} quietly presumes that standard input
is the file source which will
specify the machine ClassAd that includes the public IP address
and subnet of the machine.

\Condor{power} needs a complete specification of the machine to
be successful.
If a MAC address is provided on the command line, but no subnet is given,
then the default value for the subnet is used.
If a subnet is provided on the command line, but no MAC address is given,
then \Condor{power} falls back to taking its information in the form
of the machine ClassAd as provided in a file or on standard input.
Note that this case implies that the command line specification of the 
subnet is ignored.

\Condor{power} relies on the router receiving the WOL packet to correctly
broadcast the request. 
Since routers are often configured to ignore
requests to broadcast messages on a different subnet than the sender,
the send of a WOL packet to a machine on a different subnet may fail.

\begin{Options}
  \OptItem{\Opt{-h}}{Print usage information and exit.}
  \OptItem{\Opt{-d}}{Enable debugging messages.}
  \OptItem{\Opt{-i}}{Read a ClassAd that is piped in through standard input. }
  \OptItem{\OptArg{-m}{MACaddress}} {Specify the MAC address in the standard
    format of six groups of two hexadecimal digits separated by colons. }
%  \OptItem{\OptArg{-p}{port}} {Specify a port. }
  \OptItem{\OptArg{-s}{subnet}}{Specify the subnet in the standard form
    of a mask for an IPv4 address.
    Without this option, a global broadcast will be sent. }

\end{Options}

\ExitStatus

\Condor{power} will exit with a status value of 0 (zero) upon success,
and it will exit with the value 1 (one) upon failure.

\end{ManPage}
